% Options for packages loaded elsewhere
\PassOptionsToPackage{unicode}{hyperref}
\PassOptionsToPackage{hyphens}{url}
%
\documentclass[
]{article}
\title{2: Reproducibility and Coding Basics CEO}
\author{Environmental Data Analytics \textbar{} John Fay and Luana Lima
\textbar{} Developed by Kateri Salk}
\date{Spring 2022}

\usepackage{amsmath,amssymb}
\usepackage{lmodern}
\usepackage{iftex}
\ifPDFTeX
  \usepackage[T1]{fontenc}
  \usepackage[utf8]{inputenc}
  \usepackage{textcomp} % provide euro and other symbols
\else % if luatex or xetex
  \usepackage{unicode-math}
  \defaultfontfeatures{Scale=MatchLowercase}
  \defaultfontfeatures[\rmfamily]{Ligatures=TeX,Scale=1}
\fi
% Use upquote if available, for straight quotes in verbatim environments
\IfFileExists{upquote.sty}{\usepackage{upquote}}{}
\IfFileExists{microtype.sty}{% use microtype if available
  \usepackage[]{microtype}
  \UseMicrotypeSet[protrusion]{basicmath} % disable protrusion for tt fonts
}{}
\makeatletter
\@ifundefined{KOMAClassName}{% if non-KOMA class
  \IfFileExists{parskip.sty}{%
    \usepackage{parskip}
  }{% else
    \setlength{\parindent}{0pt}
    \setlength{\parskip}{6pt plus 2pt minus 1pt}}
}{% if KOMA class
  \KOMAoptions{parskip=half}}
\makeatother
\usepackage{xcolor}
\IfFileExists{xurl.sty}{\usepackage{xurl}}{} % add URL line breaks if available
\IfFileExists{bookmark.sty}{\usepackage{bookmark}}{\usepackage{hyperref}}
\hypersetup{
  pdftitle={2: Reproducibility and Coding Basics CEO},
  pdfauthor={Environmental Data Analytics \textbar{} John Fay and Luana Lima \textbar{} Developed by Kateri Salk},
  hidelinks,
  pdfcreator={LaTeX via pandoc}}
\urlstyle{same} % disable monospaced font for URLs
\usepackage[margin=2.54cm]{geometry}
\usepackage{color}
\usepackage{fancyvrb}
\newcommand{\VerbBar}{|}
\newcommand{\VERB}{\Verb[commandchars=\\\{\}]}
\DefineVerbatimEnvironment{Highlighting}{Verbatim}{commandchars=\\\{\}}
% Add ',fontsize=\small' for more characters per line
\usepackage{framed}
\definecolor{shadecolor}{RGB}{248,248,248}
\newenvironment{Shaded}{\begin{snugshade}}{\end{snugshade}}
\newcommand{\AlertTok}[1]{\textcolor[rgb]{0.94,0.16,0.16}{#1}}
\newcommand{\AnnotationTok}[1]{\textcolor[rgb]{0.56,0.35,0.01}{\textbf{\textit{#1}}}}
\newcommand{\AttributeTok}[1]{\textcolor[rgb]{0.77,0.63,0.00}{#1}}
\newcommand{\BaseNTok}[1]{\textcolor[rgb]{0.00,0.00,0.81}{#1}}
\newcommand{\BuiltInTok}[1]{#1}
\newcommand{\CharTok}[1]{\textcolor[rgb]{0.31,0.60,0.02}{#1}}
\newcommand{\CommentTok}[1]{\textcolor[rgb]{0.56,0.35,0.01}{\textit{#1}}}
\newcommand{\CommentVarTok}[1]{\textcolor[rgb]{0.56,0.35,0.01}{\textbf{\textit{#1}}}}
\newcommand{\ConstantTok}[1]{\textcolor[rgb]{0.00,0.00,0.00}{#1}}
\newcommand{\ControlFlowTok}[1]{\textcolor[rgb]{0.13,0.29,0.53}{\textbf{#1}}}
\newcommand{\DataTypeTok}[1]{\textcolor[rgb]{0.13,0.29,0.53}{#1}}
\newcommand{\DecValTok}[1]{\textcolor[rgb]{0.00,0.00,0.81}{#1}}
\newcommand{\DocumentationTok}[1]{\textcolor[rgb]{0.56,0.35,0.01}{\textbf{\textit{#1}}}}
\newcommand{\ErrorTok}[1]{\textcolor[rgb]{0.64,0.00,0.00}{\textbf{#1}}}
\newcommand{\ExtensionTok}[1]{#1}
\newcommand{\FloatTok}[1]{\textcolor[rgb]{0.00,0.00,0.81}{#1}}
\newcommand{\FunctionTok}[1]{\textcolor[rgb]{0.00,0.00,0.00}{#1}}
\newcommand{\ImportTok}[1]{#1}
\newcommand{\InformationTok}[1]{\textcolor[rgb]{0.56,0.35,0.01}{\textbf{\textit{#1}}}}
\newcommand{\KeywordTok}[1]{\textcolor[rgb]{0.13,0.29,0.53}{\textbf{#1}}}
\newcommand{\NormalTok}[1]{#1}
\newcommand{\OperatorTok}[1]{\textcolor[rgb]{0.81,0.36,0.00}{\textbf{#1}}}
\newcommand{\OtherTok}[1]{\textcolor[rgb]{0.56,0.35,0.01}{#1}}
\newcommand{\PreprocessorTok}[1]{\textcolor[rgb]{0.56,0.35,0.01}{\textit{#1}}}
\newcommand{\RegionMarkerTok}[1]{#1}
\newcommand{\SpecialCharTok}[1]{\textcolor[rgb]{0.00,0.00,0.00}{#1}}
\newcommand{\SpecialStringTok}[1]{\textcolor[rgb]{0.31,0.60,0.02}{#1}}
\newcommand{\StringTok}[1]{\textcolor[rgb]{0.31,0.60,0.02}{#1}}
\newcommand{\VariableTok}[1]{\textcolor[rgb]{0.00,0.00,0.00}{#1}}
\newcommand{\VerbatimStringTok}[1]{\textcolor[rgb]{0.31,0.60,0.02}{#1}}
\newcommand{\WarningTok}[1]{\textcolor[rgb]{0.56,0.35,0.01}{\textbf{\textit{#1}}}}
\usepackage{graphicx}
\makeatletter
\def\maxwidth{\ifdim\Gin@nat@width>\linewidth\linewidth\else\Gin@nat@width\fi}
\def\maxheight{\ifdim\Gin@nat@height>\textheight\textheight\else\Gin@nat@height\fi}
\makeatother
% Scale images if necessary, so that they will not overflow the page
% margins by default, and it is still possible to overwrite the defaults
% using explicit options in \includegraphics[width, height, ...]{}
\setkeys{Gin}{width=\maxwidth,height=\maxheight,keepaspectratio}
% Set default figure placement to htbp
\makeatletter
\def\fps@figure{htbp}
\makeatother
\setlength{\emergencystretch}{3em} % prevent overfull lines
\providecommand{\tightlist}{%
  \setlength{\itemsep}{0pt}\setlength{\parskip}{0pt}}
\setcounter{secnumdepth}{-\maxdimen} % remove section numbering
\ifLuaTeX
  \usepackage{selnolig}  % disable illegal ligatures
\fi

\begin{document}
\maketitle

\hypertarget{objectives}{%
\subsection{Objectives}\label{objectives}}

\begin{enumerate}
\def\labelenumi{\arabic{enumi}.}
\tightlist
\item
  Discuss the benefits and approach for reproducible data analysis
\item
  Perform simple operations using R coding syntax
\item
  Call and create functions in R
\end{enumerate}

\hypertarget{reproducible-data-analysis}{%
\subsection{Reproducible Data
Analysis}\label{reproducible-data-analysis}}

\hypertarget{fundamentals-of-reproducibility}{%
\subsubsection{Fundamentals of
reproducibility}\label{fundamentals-of-reproducibility}}

\textbf{Reproducibility}: when someone else (e.g., future self) can
obtain the same outcomes from the same dataset and analysis

\begin{itemize}
\tightlist
\item
  Raw data are always separate from processed data
\item
  Link data transformations with a reproducible pipeline
\item
  Raw datasets NEVER changed
\item
  Cleaning/transformations done through coding, not by editing within
  Excel
\item
  Edits documented by well-commented code
\item
  Majority of time spent in the data processing phase (clean, wrangle)
\end{itemize}

\hypertarget{rules-and-conventions}{%
\subsubsection{Rules and Conventions}\label{rules-and-conventions}}

\begin{itemize}
\tightlist
\item
  Data stored in nonproprietary software (e.g., .csv, .md, .txt)
\item
  File names in ASCII text aka common characters
\item
  No spaces in file names!
\item
  Consistent file naming conventions, be specific with the names
\item
  Store data, code, and output in separate folders within the project
  folder because later when you go back to a project you want to know
  what is what
\end{itemize}

\hypertarget{version-control}{%
\subsubsection{Version Control}\label{version-control}}

This semester, we will incorporate the fundamentals of \textbf{version
control}, the process by which all changes to code, text, and files are
tracked. In this manner, we're also able to maintain data and
information to support collaborative projects, but to also make sure
your analyses are preserved.

Before coming to class, you were asked to create a GitHub.com account.
\textbf{GitHub} is the web hosting platform for maintaining our Git
repositories. Our version control system for the purposes of this course
is \textbf{Git}.

\hypertarget{rstudio-basics}{%
\subsection{RStudio Basics}\label{rstudio-basics}}

Welcome to the RStudio interface. When you open RStudio, you will see
four panels:

\textbf{Source Code Editor} (top left) includes a tab structure to pull
up and edit R scripts and markdown documents.

\textbf{Console} (bottom left) interacts with R processes. R code is run
here. There is also a tab here called \texttt{Terminal} which will allow
you to access git functionality.

\textbf{Workspace Browser} (top right) holds the
\texttt{global\ environment} that is populated by analyses run in each R
session. There is also a \texttt{history} tab and a \texttt{git} tab.

\textbf{Notebook} (bottom right) holds tabs for \texttt{files},
\texttt{plots}, \texttt{packages}, and \texttt{help}. You will interact
with each of these functionalities, and we will explain each as they
come up.

More on the functionality of each of these panels as we move through
this lesson.

\hypertarget{rmarkdown-documents}{%
\subsection{RMarkdown documents}\label{rmarkdown-documents}}

You are currently viewing an R Markdown document. This type of file
includes text chunks and R code chunks that can be viewed together. R
Markdown documents can also be ``knitted'' into a PDF or html format
(more on this later).

An R script file is similar to an R Markdown document, except it
\emph{only} includes R code. Any text that is included in an R script
that it not intended to be run as R code must be ``commented out'' so
that R does not interpret the text as code. We will practice with R
scripts later.

You may also choose to type or paste R code directly into the console.
This is not a recommended method, as it undermines the goals of
reproducibility (code is not saved). However, typing directly into the
console can be useful if you need to do something that is strictly
temporary (e.g., look at a summary of a dataset or determine the class
of a variable)

\hypertarget{r-coding-basics}{%
\subsection{R Coding basics}\label{r-coding-basics}}

\hypertarget{r-as-a-calculator}{%
\subsubsection{R as a calculator}\label{r-as-a-calculator}}

Below is a chunk of R code. You can run R code in several ways:

\begin{itemize}
\item
  Place your cursor on the line of R code that you want to run, then
  press \texttt{control\ +\ enter} (PC) or \texttt{command\ +\ enter}
  (Mac). Your R code should appear in the console, followed by any
  output generated by the code.
\item
  Highlight line(s) of R code, then press \texttt{control\ +\ enter}
  (PC) or \texttt{command\ +\ enter} (Mac). Your R code should appear in
  the console, followed by any output generated by the code. This is a
  good option if you want to run multiple lines of code at once.
\end{itemize}

\begin{Shaded}
\begin{Highlighting}[]
\CommentTok{\# Basic math}
\DecValTok{1} \SpecialCharTok{+} \DecValTok{1}
\end{Highlighting}
\end{Shaded}

\begin{verbatim}
## [1] 2
\end{verbatim}

\begin{Shaded}
\begin{Highlighting}[]
\DecValTok{1} \SpecialCharTok{{-}} \DecValTok{1}
\end{Highlighting}
\end{Shaded}

\begin{verbatim}
## [1] 0
\end{verbatim}

\begin{Shaded}
\begin{Highlighting}[]
\DecValTok{2} \SpecialCharTok{*} \DecValTok{2}
\end{Highlighting}
\end{Shaded}

\begin{verbatim}
## [1] 4
\end{verbatim}

\begin{Shaded}
\begin{Highlighting}[]
\DecValTok{1} \SpecialCharTok{/} \DecValTok{2}
\end{Highlighting}
\end{Shaded}

\begin{verbatim}
## [1] 0.5
\end{verbatim}

\begin{Shaded}
\begin{Highlighting}[]
\DecValTok{1} \SpecialCharTok{/} \DecValTok{200} \SpecialCharTok{*} \DecValTok{30}
\end{Highlighting}
\end{Shaded}

\begin{verbatim}
## [1] 0.15
\end{verbatim}

\begin{Shaded}
\begin{Highlighting}[]
\DecValTok{5} \SpecialCharTok{+} \DecValTok{2} \SpecialCharTok{*} \DecValTok{3}
\end{Highlighting}
\end{Shaded}

\begin{verbatim}
## [1] 11
\end{verbatim}

\begin{Shaded}
\begin{Highlighting}[]
\NormalTok{(}\DecValTok{5} \SpecialCharTok{+} \DecValTok{2}\NormalTok{) }\SpecialCharTok{*} \DecValTok{3}
\end{Highlighting}
\end{Shaded}

\begin{verbatim}
## [1] 21
\end{verbatim}

\begin{Shaded}
\begin{Highlighting}[]
\CommentTok{\# Common terms}
\FunctionTok{sqrt}\NormalTok{(}\DecValTok{25}\NormalTok{)}
\end{Highlighting}
\end{Shaded}

\begin{verbatim}
## [1] 5
\end{verbatim}

\begin{Shaded}
\begin{Highlighting}[]
\FunctionTok{sin}\NormalTok{(}\DecValTok{3}\NormalTok{)}
\end{Highlighting}
\end{Shaded}

\begin{verbatim}
## [1] 0.14112
\end{verbatim}

\begin{Shaded}
\begin{Highlighting}[]
\NormalTok{pi}
\end{Highlighting}
\end{Shaded}

\begin{verbatim}
## [1] 3.141593
\end{verbatim}

\begin{Shaded}
\begin{Highlighting}[]
\CommentTok{\# Summary statistics}
\FunctionTok{mean}\NormalTok{(}\DecValTok{5}\NormalTok{, }\DecValTok{4}\NormalTok{, }\DecValTok{6}\NormalTok{, }\DecValTok{4}\NormalTok{, }\DecValTok{6}\NormalTok{)}
\end{Highlighting}
\end{Shaded}

\begin{verbatim}
## [1] 5
\end{verbatim}

\begin{Shaded}
\begin{Highlighting}[]
\FunctionTok{median}\NormalTok{(}\DecValTok{5}\NormalTok{, }\DecValTok{4}\NormalTok{, }\DecValTok{6}\NormalTok{, }\DecValTok{4}\NormalTok{, }\DecValTok{6}\NormalTok{)}
\end{Highlighting}
\end{Shaded}

\begin{verbatim}
## [1] 5
\end{verbatim}

\begin{Shaded}
\begin{Highlighting}[]
\CommentTok{\# Conditional statements: output TRUE or FASLE}
\DecValTok{4} \SpecialCharTok{\textgreater{}} \DecValTok{5}
\end{Highlighting}
\end{Shaded}

\begin{verbatim}
## [1] FALSE
\end{verbatim}

\begin{Shaded}
\begin{Highlighting}[]
\DecValTok{4} \SpecialCharTok{\textless{}} \DecValTok{5}
\end{Highlighting}
\end{Shaded}

\begin{verbatim}
## [1] TRUE
\end{verbatim}

\begin{Shaded}
\begin{Highlighting}[]
\DecValTok{4} \SpecialCharTok{!=} \DecValTok{5} \CommentTok{\#different than}
\end{Highlighting}
\end{Shaded}

\begin{verbatim}
## [1] TRUE
\end{verbatim}

\begin{Shaded}
\begin{Highlighting}[]
\DecValTok{4} \SpecialCharTok{==} \DecValTok{5} \CommentTok{\#equal to}
\end{Highlighting}
\end{Shaded}

\begin{verbatim}
## [1] FALSE
\end{verbatim}

\hypertarget{objects}{%
\subsubsection{Objects}\label{objects}}

You can create R objects with an \emph{assignment} statement. The
indicator for an assignment is the \texttt{\textless{}-} symbol. A good
way to think about the meaning of an assignment statement is ``object
name (lefthand side) gets value (righthand side).''

A quick note: in many situations, a \texttt{=} sign will substitute for
a \texttt{\textless{}-}. Resist this temptation! This will be confusing
later, when \texttt{=} means something else.

\begin{Shaded}
\begin{Highlighting}[]
\NormalTok{x }\OtherTok{\textless{}{-}} \DecValTok{3}\SpecialCharTok{*}\DecValTok{4}
\end{Highlighting}
\end{Shaded}

Notice that \texttt{x} has also just shown up in your Environment tab.
Now, call up the object \texttt{x} by running x by itself. Then its
value will show up in the console.

\begin{Shaded}
\begin{Highlighting}[]
\NormalTok{x}
\end{Highlighting}
\end{Shaded}

\begin{verbatim}
## [1] 12
\end{verbatim}

\hypertarget{naming}{%
\subsubsection{Naming}\label{naming}}

R objects can be named with a combination of letters, numbers,
underscore (\texttt{\_}) and period (\texttt{.}). The best R object
names are \emph{informative}. Resist the temptation to call your R
object something convenient, like ``a'', ``b'', and so on. Calling your
R object something specific means that you can call up that object later
and have an idea of what it contains, with less need for specific
context.

Informative names are the first illustration of a common data management
recommendation: take the time to use best management practices at the
outset, and it will save you time in the long term.

Importantly, you may never call an R object ``data''. This word is
reserved for a specific function and may not be assigned as a name. To
work around this, many people call their R objects ``dat'', which is
another example of a less-than-ideal data management practice because it
is not informative.

Run the first line of code below (the assign
long\_name\_for\_illustration assignment below. Then, type in ``long''
and press \texttt{tab}. What happens? R fills in with tab-completion!

What happens if there is a typo in your code? This is just showing that
R is case sensitive. Type the following in the R window:
Long\_name\_for\_illustration longnameforillustration

\begin{Shaded}
\begin{Highlighting}[]
\NormalTok{long\_name\_for\_illustration }\OtherTok{\textless{}{-}} \DecValTok{11}
\end{Highlighting}
\end{Shaded}

\hypertarget{comments}{%
\subsubsection{Comments}\label{comments}}

Within your R code, it is often useful to include notes about your
workflow. So that these aren't interpreted by the software as code,
precede the notes with a \texttt{\#} sign. Your editor will display this
comment as a different color to indicate it will not be run in the
console. Comments can be placed on their own lines or at the end of a
line of code.

\begin{Shaded}
\begin{Highlighting}[]
\CommentTok{\# I am demonstrating a comment here. }

\DecValTok{1} \SpecialCharTok{+} \DecValTok{1} \CommentTok{\# This is a simple math problem}
\end{Highlighting}
\end{Shaded}

\begin{verbatim}
## [1] 2
\end{verbatim}

\hypertarget{functions}{%
\subsubsection{Functions}\label{functions}}

R functions are the major tool used in R. Functions can do virtually
unlimited things within the R universe, but each function requires
specific inputs that are provided under specific syntax. We will start
with a simple function that is built into R, \texttt{seq}

\begin{Shaded}
\begin{Highlighting}[]
\CommentTok{\#seq aka sequence outputs a sequence of numbers. You specify start and end numbers. }
\FunctionTok{seq}\NormalTok{(}\DecValTok{1}\NormalTok{, }\DecValTok{10}\NormalTok{)}
\end{Highlighting}
\end{Shaded}

\begin{verbatim}
##  [1]  1  2  3  4  5  6  7  8  9 10
\end{verbatim}

\begin{Shaded}
\begin{Highlighting}[]
\NormalTok{ten\_sequence }\OtherTok{\textless{}{-}} \FunctionTok{seq}\NormalTok{(}\DecValTok{1}\NormalTok{, }\DecValTok{10}\NormalTok{)}
\NormalTok{ten\_sequence}
\end{Highlighting}
\end{Shaded}

\begin{verbatim}
##  [1]  1  2  3  4  5  6  7  8  9 10
\end{verbatim}

\begin{Shaded}
\begin{Highlighting}[]
\FunctionTok{seq}\NormalTok{(}\DecValTok{1}\NormalTok{, }\DecValTok{10}\NormalTok{, }\DecValTok{2}\NormalTok{) }\CommentTok{\# from, to, by}
\end{Highlighting}
\end{Shaded}

\begin{verbatim}
## [1] 1 3 5 7 9
\end{verbatim}

\begin{Shaded}
\begin{Highlighting}[]
\CommentTok{\#outputs 1 3 5 7 9. By has a default value of 1 and doesn\textquotesingle{}t have to be specified}
\end{Highlighting}
\end{Shaded}

The basic form of a function is \texttt{functionname()}, and the
packages we will use in this class will use these basic forms. However,
there may be situations when you will want to create your own function.
Below is a description of how to write functions through the metaphor of
creating a recipe (credit: @IsabellaGhement on Twitter).

Writing a function is like writing a recipe. Your function will need a
recipe name (functionname). Your recipe ingredients will go inside the
parentheses. The recipe steps and end product go inside the curly
brackets.

\begin{Shaded}
\begin{Highlighting}[]
\NormalTok{functionname }\OtherTok{\textless{}{-}} \ControlFlowTok{function}\NormalTok{()\{}
  
\NormalTok{\}}
\end{Highlighting}
\end{Shaded}

A single ingredient recipe:

\begin{Shaded}
\begin{Highlighting}[]
\CommentTok{\# Write the recipe}
\NormalTok{recipe1 }\OtherTok{\textless{}{-}} \ControlFlowTok{function}\NormalTok{(x)\{}
\NormalTok{  mix }\OtherTok{\textless{}{-}}\NormalTok{ x}\SpecialCharTok{*}\DecValTok{2}
  \FunctionTok{return}\NormalTok{(mix)}
\NormalTok{\}}
\CommentTok{\#so this is naming the function recipe1 which you will use to run the function later. In the parentheses, x is the numbers you inout i.e. 5. In the curly brackets, you assign a name to what you want the function to do to x i.e. 5*2. and you tell R what to return as the output which should be what you name the function action in the curly brackets. The ouput will be 10.}


\CommentTok{\# Bake the recipe. Assign the function to a different object with a specific input into the function.}
\NormalTok{simplemeal }\OtherTok{\textless{}{-}} \FunctionTok{recipe1}\NormalTok{(}\DecValTok{5}\NormalTok{)}

\CommentTok{\# Serve the recipe. Run the code and you get 10!}
\NormalTok{simplemeal}
\end{Highlighting}
\end{Shaded}

\begin{verbatim}
## [1] 10
\end{verbatim}

\begin{Shaded}
\begin{Highlighting}[]
\CommentTok{\#Yes it would be easier to just do simplemeal \textless{}{-} 5*2 BUT having the function saves you typing and one line of code for many computations. When working with more complex data the value of creating your own functions becomes clearer. }
\end{Highlighting}
\end{Shaded}

Two single ingredient recipes, baked at the same time: Cat Note: you do
two things with one ingredient i.e.~x

\begin{Shaded}
\begin{Highlighting}[]
\NormalTok{recipe2 }\OtherTok{\textless{}{-}} \ControlFlowTok{function}\NormalTok{(x)\{}
\NormalTok{  mix1 }\OtherTok{\textless{}{-}}\NormalTok{ x}\SpecialCharTok{*}\DecValTok{2}
\NormalTok{  mix2 }\OtherTok{\textless{}{-}}\NormalTok{ x}\SpecialCharTok{/}\DecValTok{2}
  \FunctionTok{return}\NormalTok{(}\FunctionTok{list}\NormalTok{(}\AttributeTok{mix1 =}\NormalTok{ mix1, }\CommentTok{\#comma indicates we continue onto the next line}
              \AttributeTok{mix2 =}\NormalTok{ mix2))}
\NormalTok{  \}}

\CommentTok{\#A list is a vector with more than one object. The list from code above will have two objects. Have mix1 = mix1 so that objects in the list are called the same as the objects in the function, but could name them anything.}

\NormalTok{doublesimplemeal }\OtherTok{\textless{}{-}} \FunctionTok{recipe2}\NormalTok{(}\DecValTok{6}\NormalTok{)}
\NormalTok{doublesimplemeal}
\end{Highlighting}
\end{Shaded}

\begin{verbatim}
## $mix1
## [1] 12
## 
## $mix2
## [1] 3
\end{verbatim}

\begin{Shaded}
\begin{Highlighting}[]
\CommentTok{\#output is $mix1 [1] 12 $mix2 [1] 3}
\CommentTok{\#the $ sign let\textquotesingle{}s you look for objects in the list or just call part of the list}
\CommentTok{\#If don\textquotesingle{}t know the name of the object in the list just type the function name with $ after and R will show you options for your list.}

\NormalTok{doublesimplemeal}\SpecialCharTok{$}\NormalTok{mix1}
\end{Highlighting}
\end{Shaded}

\begin{verbatim}
## [1] 12
\end{verbatim}

\begin{Shaded}
\begin{Highlighting}[]
\CommentTok{\#output [1] 12}

\CommentTok{\#the [1] is just showing that it\textquotesingle{}s in the 1 spot in the vector}
\end{Highlighting}
\end{Shaded}

Two double ingredient recipes, baked at the same time: Cat Note: adding
ingredients! So you need to provide two inputs in this example.

\begin{Shaded}
\begin{Highlighting}[]
\NormalTok{recipe3 }\OtherTok{\textless{}{-}} \ControlFlowTok{function}\NormalTok{(x, f)\{}
\NormalTok{  mix1 }\OtherTok{\textless{}{-}}\NormalTok{ x}\SpecialCharTok{*}\NormalTok{f}
\NormalTok{  mix2 }\OtherTok{\textless{}{-}}\NormalTok{ x}\SpecialCharTok{/}\NormalTok{f}
  \FunctionTok{return}\NormalTok{(}\FunctionTok{list}\NormalTok{(}\AttributeTok{mix1 =}\NormalTok{ mix1, }\CommentTok{\#comma indicates we continue onto the next line}
              \AttributeTok{mix2 =}\NormalTok{ mix2))}
\NormalTok{\}}

\NormalTok{doublecomplexmeal }\OtherTok{\textless{}{-}} \FunctionTok{recipe3}\NormalTok{(}\AttributeTok{x =} \DecValTok{5}\NormalTok{, }\AttributeTok{f =} \DecValTok{2}\NormalTok{)}
\NormalTok{doublecomplexmeal}
\end{Highlighting}
\end{Shaded}

\begin{verbatim}
## $mix1
## [1] 10
## 
## $mix2
## [1] 2.5
\end{verbatim}

\begin{Shaded}
\begin{Highlighting}[]
\NormalTok{doublecomplexmeal}\SpecialCharTok{$}\NormalTok{mix1}
\end{Highlighting}
\end{Shaded}

\begin{verbatim}
## [1] 10
\end{verbatim}

Make a recipe based on the ingredients you have

\begin{Shaded}
\begin{Highlighting}[]
\CommentTok{\#For this example it\textquotesingle{}s like IF you have less than 3 carrots (input less than 3) you eat carrot sticks or ELSE you have 3 or more carrots (input 3 or greater) you can make soup. You do one or the other actions. }

\NormalTok{recipe4 }\OtherTok{\textless{}{-}} \ControlFlowTok{function}\NormalTok{(x) \{}
  \ControlFlowTok{if}\NormalTok{(x }\SpecialCharTok{\textless{}} \DecValTok{3}\NormalTok{) \{}
\NormalTok{    x}\SpecialCharTok{*}\DecValTok{2}
\NormalTok{  \} }
  \ControlFlowTok{else}\NormalTok{ \{}
\NormalTok{    x}\SpecialCharTok{/}\DecValTok{2}
\NormalTok{  \}}
\NormalTok{\}}

\CommentTok{\#Just put x if you want it to be equal to x}

\NormalTok{recipe5 }\OtherTok{\textless{}{-}} \ControlFlowTok{function}\NormalTok{(x) \{}
  \ControlFlowTok{if}\NormalTok{(x }\SpecialCharTok{\textless{}} \DecValTok{3}\NormalTok{) \{}
\NormalTok{    x}\SpecialCharTok{*}\DecValTok{2}
\NormalTok{  \} }
  \ControlFlowTok{else} \ControlFlowTok{if}\NormalTok{ (x }\SpecialCharTok{\textgreater{}} \DecValTok{3}\NormalTok{) \{}
\NormalTok{    x}\SpecialCharTok{/}\DecValTok{2}
\NormalTok{  \}}
  \ControlFlowTok{else}\NormalTok{ \{}
\NormalTok{    x}
\NormalTok{  \}}
\NormalTok{\}}
\NormalTok{meal }\OtherTok{\textless{}{-}} \FunctionTok{recipe4}\NormalTok{(}\DecValTok{4}\NormalTok{); meal}
\end{Highlighting}
\end{Shaded}

\begin{verbatim}
## [1] 2
\end{verbatim}

\begin{Shaded}
\begin{Highlighting}[]
\NormalTok{meal2 }\OtherTok{\textless{}{-}} \FunctionTok{recipe4}\NormalTok{(}\DecValTok{2}\NormalTok{); meal2}
\end{Highlighting}
\end{Shaded}

\begin{verbatim}
## [1] 4
\end{verbatim}

\begin{Shaded}
\begin{Highlighting}[]
\NormalTok{meal3 }\OtherTok{\textless{}{-}} \FunctionTok{recipe5}\NormalTok{(}\DecValTok{3}\NormalTok{); meal3}
\end{Highlighting}
\end{Shaded}

\begin{verbatim}
## [1] 3
\end{verbatim}

\begin{Shaded}
\begin{Highlighting}[]
\CommentTok{\#the above code assigns name meal to recipe4 when 4 is the input and the semicolon followed by the assigned name will then return the output without you having to run meal as a new line of code to get the output after assignment}

\CommentTok{\#below code has same output of recipe4 but simplified. It uses a function within a function \#funcception. The things within the parentheses within the curly brackets are called elements.}
\NormalTok{recipe6 }\OtherTok{\textless{}{-}} \ControlFlowTok{function}\NormalTok{(x)\{}
  \FunctionTok{ifelse}\NormalTok{(x}\SpecialCharTok{\textless{}}\DecValTok{3}\NormalTok{, x}\SpecialCharTok{*}\DecValTok{2}\NormalTok{, x}\SpecialCharTok{/}\DecValTok{2}\NormalTok{) }\CommentTok{\#logical\_expression, if TRUE (if the input is less than 3), if FALSE (if the input is not less than 3)}
  
\NormalTok{\}}

\NormalTok{meal4 }\OtherTok{\textless{}{-}} \FunctionTok{recipe6}\NormalTok{(}\DecValTok{4}\NormalTok{); meal4}
\end{Highlighting}
\end{Shaded}

\begin{verbatim}
## [1] 2
\end{verbatim}

\begin{Shaded}
\begin{Highlighting}[]
\NormalTok{meal5 }\OtherTok{\textless{}{-}} \FunctionTok{recipe6}\NormalTok{(}\DecValTok{2}\NormalTok{); meal5}
\end{Highlighting}
\end{Shaded}

\begin{verbatim}
## [1] 4
\end{verbatim}

\hypertarget{getting-help-within-r}{%
\subsubsection{Getting help within R}\label{getting-help-within-r}}

In many ways, the help functionality in R is limited by the fact that
you need to have a good understanding of specific functions for the help
to be useful. Google and Stack Overflow are often more helpful than the
help within R. We will practice those skills later.

For now, here are some ways to access the help tools in R:

\begin{itemize}
\item
  Within your R chunks in your editor, type in \texttt{??function}. This
  will bring up the help pane in the notebook, which you can then
  navigate through to find what you need.
\item
  In the console, type in \texttt{help(function)}. This will bring up
  the help pane in the notebook at the page for that function.
\item
  Navigate to the help pane in the notebook and type the function into
  the search bar.
\end{itemize}

\begin{Shaded}
\begin{Highlighting}[]
\NormalTok{??seq}
\end{Highlighting}
\end{Shaded}

\begin{verbatim}
## starting httpd help server ... done
\end{verbatim}

\hypertarget{tips-and-tricks}{%
\subsection{Tips and Tricks}\label{tips-and-tricks}}

\begin{itemize}
\item
  Spaces (generally) don't matter. One notable exception is that spaces
  within quotation marks \emph{do} matter. Because quotations mean you
  are adding text to an output
\item
  Case matters
\item
  Parentheses and quotation marks appear in pairs when typed into
  RStudio.
\item
  When typing long names, use the \texttt{tab} key partway through the
  name to generate autocomplete options.
\item
  In the upper right corner of the editor is a button with multiple
  horizontal lines. Clicking this button will bring up the outline of
  the document. Headings in the outline are determined by how you've
  defined them in the document.
\end{itemize}

R Markdown and R script files can be organized into sections. Sections
can be expanded or collapsed as desired via three options:

\begin{itemize}
\item
  On the left hand side of the editor, you will see arrows to the right
  of the line numbers. Clicking on these arrows will collapse or expand
  the section. When a section is collapsed, a double-arrow box will
  appear within the script. You can also click on this box directly to
  expand the section. This cleans up your code so that a long doc is
  cleaner but the commands don't go away just are hidden
\item
  Navigating from the menu bar, the Edit menu will bring you to the
  option ``Folding''. Gives you the option to collapse all and then you
  get a kind of table of contents for your markdown file with the titles
  you gave things the (words next to the \# that are outside of code
  blocks). This option can be especially helpful when you first open a
  file and decide if you want to navigate between sections or run
  sections sequentially.
\item
  Highlight a section of text, and then press \texttt{option} +
  \texttt{command} + \texttt{L} (Mac) or \texttt{alt} + \texttt{L}
  (Windows). Add the \texttt{shift} key to this combination to expand,
  or click the box.
\end{itemize}

\end{document}
